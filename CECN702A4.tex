\documentclass{article}
\usepackage{graphicx}
\usepackage{setspace}
\usepackage{fullpage}
\usepackage{changepage}
\usepackage{array}
\usepackage{multirow}
\usepackage{multicol}
\usepackage[english]{babel}
\usepackage[latin1]{inputenc}
\usepackage[autostyle=true]{csquotes}

%%Packages for math
\usepackage{mathtools,amsthm,amssymb,amsfonts}

%%Packages for graphs
\usepackage{pgf}
\usepackage{tikz}
\usetikzlibrary{arrows,automata}


%%Packages for code
\usepackage{spverbatim}


%%Pure laziness.
\newcommand{\N}{\mathbb{N}}
\newcommand{\Z}{\mathbb{Z}}
\newcommand{\R}{\mathbb{R}}
\newcommand{\Q}{\mathbb{Q}}
\newcommand{\C}{\mathbb{C}}
\newcommand{\matr}[1]{#1}
\newcommand{\overbar}[1]{\mkern 1.5mu\overline{\mkern-1.5mu#1\mkern-1.5mu}\mkern 1.5mu}
\DeclareMathOperator\cis{cis}%cis
\DeclareMathOperator{\E}{\mathbb{E}}% expected value
\MakeOuterQuote{"}



% TITLE PAGE GUBBINS
%----------------------------------------------------------------------
\newcommand{\thetitle}{Assignment 4}%Change title here
\newcommand{\coursecode}{CECN702}%Change course code here
\newcommand{\thedate}{\today}%Change the due date here


\begin{document}
% Title Page
%----------------------------------------------------------------------
\input{titlepage}
%----------------------------------------------------------------------


% Start Assignment
%----------------------------------------------------------------------
\doublespacing
\section*{1}
\label{sec:1.1}


\begin{singlespace}
	\begin{spverbatim}
		> data <- read.csv("CRSP.csv", header = TRUE)
		> attach(data)
		> y <- acf(ExReturn, plot = FALSE)
		
		> y$acf[2:13]
		[1]  0.052002799 -0.048954063  0.004106913 -0.016362529  0.089864263 -0.049824430
		[7] -0.035829748 -0.064345437  0.013477192  0.015854456 -0.007585337  0.034117700
	\end{spverbatim}
\end{singlespace}

Since the first autocorrelation is with itself, we discard it, and take values autocorrelation object outputs 2 to 13.

\section*{2}

Suppose our model is 
\begin{singlespace}
\begin{align*}
\text{ExReturn}_{t} = \alpha + \beta \text{ExReturn}_{t-1} + U_{t}
\end{align*}

	\begin{spverbatim}
		> T <- length(ExReturn)
		> ExReturnlag <- ExReturn[1:(T-1)]
		> ExReturn <- ExReturn[2:T]
		> AR <- lm(ExReturn~ExReturnlag)
		> coef(summary(AR))
		(Intercept) ExReturnlag 
		0.3258526   0.0521232 
	\end{spverbatim}
\end{singlespace}

So our estimated model is
\begin{singlespace}
\begin{align*}
\text{ExReturn}_{t} = 0.3258526 + 0.0521232\times \text{ExReturn}_{t-1}
\end{align*}
\end{singlespace}
\section*{3}
Suppose our model is 

\begin{singlespace}
\begin{align*}
\text{ExReturn}_{t} = \alpha + \beta \text{ExReturn}_{t-1} + \gamma \text{ExReturn}_{t-1} + U_{t}
\end{align*}

	\begin{spverbatim}
		> DivYieldlag <- DivYield[1:(T-1)]
		> DivYield <- DivYield[2:T]
		> DivYieldDiff <- DivYield - DivYieldlag
		> ARDL <- lm(ExReturn ~ ExReturnlag+ DivYieldlag)
		> coef(ARDL)
		(Intercept) ExReturnlag DivYieldlag 
		3.90933002  0.05681810  0.01036858 
	\end{spverbatim}
\end{singlespace}

So our estimated model is
\begin{singlespace}
\begin{align*}
\text{ExReturn}_{t} = 3.90933002 + 0.05681810\times \text{ExReturn}_{t-1} +  0.01036858\times \text{ExReturn}_{t-1}
\end{align*}
\end{singlespace}

\section*{4}

Consider our previous AR(1) model, we text ExReturn for a unit root using the nonaugmented Dickey-Fuller test statistic. Ie, $\beta = 1$.

We can rewrite the above as 

\begin{singlespace}
\begin{align*}
\Delta \text{ExReturn}_{t} =  \alpha + \delta\text{ExReturn}_{t-1} + U_{t}
\end{align*}
\end{singlespace}


And test, 

\begin{singlespace}
\begin{align*}
\text{H}_0:\delta&= 0\\
	\text{H}_1:\delta&< 0
\end{align*}


	\begin{spverbatim}
		> ExReturnDiff <- ExReturn - ExReturnlag
		> DF <- lm(ExReturnDiff ~ ExReturnlag)
		> coef(summary(DF))["ExReturnlag", c("t value","Pr(>|t|)")]
		      t value      Pr(>|t|) 
		-2.153557e+01  9.324383e-74 
	\end{spverbatim}
\end{singlespace}

We observe that $\tau << -2.86 $ with an included constant, so we reject $\text{H}_0$











\end{document}
